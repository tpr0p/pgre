\documentclass[avery5388, frame, grid]{flashcards}

\input{preamble}

\begin{document}

%% MECHANICS
\begin{flashcard}{kinematic equations}
  {\begin{align*}
    \bar{v} &= \frac{1}{2} (v_{i} + v_{f})\\
    \Delta v &= a \Delta t + v_{i}\\
    x(t) &= \frac{1}{2} a t^{2} + v_{i} t + x_{i}\\
    v_{f}^{2} - v_{i}^{2} &= 2 a \Delta x\\
  \end{align*}}
\end{flashcard}

\begin{flashcard}{uniform circular motion}
  {\begin{align*}
      a &= \frac{v^{2}}{r}\\
  \end{align*}}
\end{flashcard}

\begin{flashcard}{kintec and potential energy}
  {
    \begin{align*}
      E_{\textrm{trans}} &= \frac{1}{2} m v^{2}\\
      E_{\textrm{rot}} &= \frac{1}{2} I \omega^{2}\\
      E_{\textrm{gpe}} &= m g h\\
      E_{\textrm{sho}} &= \frac{1}{2} k x^{2}\\
    \end{align*}
  }
\end{flashcard}

\begin{flashcard}{potential energy, force, work}
  {
    \begin{align*}
       dU &= -\vec{F} d\vec{l}\\
      \vec{F} &= - \vec{\nabla} U\\
      dW &= \vec{F} \cdot d\vec{l}\\
    \end{align*}
  }
\end{flashcard}

\begin{flashcard}{gravitational force}
  {
    \begin{align*}
      \vec{F} &= - G m_{1} m_{2} r^{-2} \hat{r_{12}}\\
    \end{align*}
  }
\end{flashcard}

\begin{flashcard}{angular momentum}
  {
    \begin{align*}
      \vec{L} &= \vec{r} \times \vec{p} = I \vec{\omega}\\
    \end{align*}
  }
\end{flashcard}

\begin{flashcard}{torque}
  {
    \begin{align*}
      \tau = \vec{r} \times \vec{F} = \frac{d \vec{L}}{dt}
    \end{align*}
  }
\end{flashcard}

\begin{flashcard}{constant angular velocity}
  {
    \begin{align*}
      \vec{F} = - m \Omega^{2} r \hat{r}_{\textrm{centrifuge}} - 2 m \vec{\Omega} \times \vec{v}
    \end{align*}
  }
\end{flashcard}

\begin{flashcard}{spherical elements}
  {
    \begin{align*}
      d\vec{l} &= dr \hat{r} + r d\theta \hat{\theta} + r \sin\theta d\phi \hat{\phi}\\
      d\vec{S} &= r^{2}\sin\theta d\theta d\phi \hat{r} + r \sin\theta dr d\phi \hat{\theta} + r dr d\theta \hat{\phi}\\
      dV &= r^{2}\sin\theta dr d\theta d\phi\\
    \end{align*}
  }
\end{flashcard}

\begin{flashcard}{moment of inertia}
  {
    \begin{align*}
      I &= \int{s^{2} dm}\\
      I &= I_{\textrm{CM}} + M s^{2}\\
      I_{z} &= I_{x} + I_{y} \; \textrm{for $z$ perpendicular to the body lying in the $x-y$ plane}\\
    \end{align*}
  }
\end{flashcard}

\begin{flashcard}{Lagrangian}
  {
    \begin{align*}
        L(\dot{q}_{i}, q_{i}, t) &= T - U\\
        \frac{d}{dt} \frac{\partial L}{\partial \dot{q}_{i}} &= \frac{\partial L}{\partial q_{i}}\\
        p_{i} &= \frac{\partial L}{\partial \dot{q}_{i}}\\
    \end{align*}
  }
\end{flashcard}

\begin{flashcard}{Hamiltonian}
  {
    \begin{align*}
      H(p, q) &= \sum_{i} p_{i} \dot{q}_{i} - L\\
      H &= T + U \; \textrm{if $U$ is independent of $\dot{q}$ and $t$}\\
      \dot{p} &= - \frac{\partial H}{\partial q}\\
      \dot{q} &= \frac{\partial H}{\partial p}\\
    \end{align*}
  }
\end{flashcard}

\begin{flashcard}{effective orbital potential}
  {
    \begin{align*}
      E = T + V &= \frac{1}{2} m \dot{r}^{2} + \frac{l^2}{2 m r^{2}} + U(r)\\
      l &= m r^{2} \dot{\phi}\\
      \mu &= \frac{m_{1} m_{2}}{m_{1} + m_{2}}\\
      E &> 0 \implies \textrm{hyperbolic}\\
      E &= 0 \implies \textrm{parabolic}\\
      E &< 0 \implies \textrm{elliptical}\\
    \end{align*}
  }
\end{flashcard}

\begin{flashcard}{Kepler's laws}
  {
    \begin{itemize}
    \item Planets move in elliptical orbits with one focus at the sun
    \item Equal orbital area sweeps out equal orbital time
    \item $T = ka^{\frac{3}{2}}$ for all planets
    \end{itemize}
  }
\end{flashcard}

\begin{flashcard}{damped oscillation}
  {
    \begin{align*}
      0 &= m\ddot{x} + b\dot{x} + kx \\
      \beta &= \frac{b}{2m}\\
      \omega_{0} &= (\frac{k}{m})^{\frac{1}{2}}\\
    \end{align*}
  }
  {
    \begin{itemize}
    \item $\beta^{2} > \omega_{0}^{2}$: overdamped decays exponentially
    \item $\beta^{2} = \omega_{0}^{2}$: critically damed
    \item $\beta^{2} < \omega_{0}^{2}$: underdamped decays exponentially, modulated by a sinusoid at $\omega_{1}^{2} = \omega_{0}^{2} - \beta^{2}$
    \end{itemize}
  }
\end{flashcard}

\begin{flashcard}{driven oscillation}
  {
    \begin{align*}
      \omega_{\textrm{R}} &= (\omega_{0}^{2} - 2 \beta^{2})^{\frac{1}{2}}\\
      D &\propto {\vert \omega_{0}^{2} - \omega^{2} \vert}^{-1} \; \textrm{for undamped, driven oscillator}\\
    \end{align*}
  }
\end{flashcard}

\begin{flashcard}{oscillators}
  {
    \begin{itemize}
    \item pendulum $\omega = (\frac{g}{l})^{\frac{1}{2}}$
    \item physical pendulum $\omega = (\frac{mgR}{I})^{\frac{1}{2}}$
    \end{itemize}
  }
\end{flashcard}

\begin{flashcard}{fluid dynamics}
  {
    \begin{align*}
      \frac{v^2}{2} &+ g z + \frac{p}{\rho} = \textrm{const}\\
      p \; \textrm{pressure, } &\rho \; \textrm{fluid density, } z \; \textrm{height of a point along streamline}\\
      F & = \rho V g\\
    \end{align*}
  }
\end{flashcard}

%% EM
\begin{flashcard}{Maxwell's equations}
  {
    \begin{align*}
      \vec{\nabla} \cdot \vec{E} &= \frac{\rho}{\epsilon_{0}}\\
      \vec{\nabla} \times \vec{E} &= \frac{- \partial \vec{B}}{\partial t}\\
      \vec{\nabla} \cdot \vec{B} &= 0\\
      \vec{\nabla} \times \vec{B} &= \mu_{0} \vec{J} + \mu_{0} \epsilon_{0} \frac{\partial \vec{E}}{\partial t}\\
      \frac{\partial \rho}{\partial t} &+ \vec{\nabla} \cdot \vec{J} = 0\\
      \vec{E} &= - \vec{\nabla} \phi - \frac{\partial \vec{A}}{\partial t}\\
      \vec{B} &= \vec{\nabla} \times \vec{A}\\
    \end{align*}
  }
\end{flashcard}

\begin{flashcard}{electrostatics}
  {
    \begin{align*}
      \vec{E} &= -\vec{\nabla} \phi \implies \phi = - \int_{a}^{b} \vec{E} \cdot d\vec{l}\\
      \nabla^{2} \phi &= - \frac{\rho}{\epsilon_{0}} \implies
      \phi = \frac{1}{4 \pi \epsilon_{0}} \int \frac{\rho(\vec{r}^{'})}{\lvert \vec{r} - \vec{r}^{'}\lvert} d^{3}\vec{r}^{'}\\
      \int_{S} \vec{E}(\vec{r}) \cdot d\vec{S} &= \frac{Q_{\textrm{enc}}}{\epsilon_{0}}\\
      \int_{C} \vec{E}(\vec{r}) \cdot d\vec{l} &= 0\\
    \end{align*}
  }
\end{flashcard}

\begin{flashcard}{boundary conditions}
  {
    Electrostatics
    \begin{align*}
      E^{\parallel}_{\textrm{out}} - E^{\parallel}_{\textrm{in}} &= 0\\
      E^{\bot}_{\textrm{out}} - E^{\bot}_{\textrm{in}} &= \frac{\sigma}{\epsilon_{0}}\\
      \phi &\textrm{is continuous}\\
      \partial \phi & \textrm{continuous when no surface charge is present}\\
    \end{align*}
    Magnetostatics
    \begin{align*}
      B^{\parallel}_{\textrm{out}} - B^{\parallel}_{\textrm{in}} &= \mu_{0} \vec{k} \times \hat{n}
      B^{\bot}_{\textrm{out}} - B^{\bot}_{\textrm{in}} &= 0
    \end{align*}
    $\vec{k}$ is the surface current density
  }
\end{flashcard}

\begin{flashcard}{conductors}
  {
    \begin{align*}
      \phi &= \textrm{const. throughout conductor}\\
      & \implies \textrm{E-field inside is zero}\\
      & \implies Q_{\textrm{inside}} = 0\\
      & \implies Q_{\textrm{net}} \textrm{confined to surface}\\
      & \implies \textrm{E-field outside is perpendicular to surface}\\
    \end{align*}
  }
\end{flashcard}

\begin{flashcard}{work and energy in (electro/magneto)statics}
  {
    \begin{align*}
      W &= \frac{1}{2} \int \rho \phi d^{3}r\\
      U_{E} &= \frac{\epsilon_{0}}{2} \int {\lvert \vec{E} \lvert}^{2} d^{3}r\\
      U_{B} &= \frac{1}{2 \mu_{0}} \int {\lvert \vec{B} \lvert}^{2} d^{3}r\\
    \end{align*}
  }
\end{flashcard}

\begin{flashcard}{capacitors}
  {
    \begin{align*}
      Q &= C \phi\\
      C_{\textrm{parallel plate}} &= \frac{\epsilon_{0} A}{d}\\
      U &= \frac{1}{2} Q^{2} C^{-1} = \frac{1}{2} C \phi^{2}\\
    \end{align*}
  }
\end{flashcard}

\begin{flashcard}{magnetostatics}
  {
    \begin{align*}
      \int_{S} \vec{B} \cdot d\vec{s} &= 0\\
      \int_{C} \vec{B} \cdot d\vec{l} &= \mu_{0} I_{\textrm{enc}}\\
      \vec{F_{B}} &= q\vec{v} \times \vec{B}\\
      \vec{B}(\vec{r}) &= \frac{\mu_{0} I}{4 \pi} \int \frac{d\vec{l} \times \vec{r^{'}}}{{\lvert \vec{r^{'}}\lvert}^{3}}\\
    \end{align*}
  }
\end{flashcard}

\begin{flashcard}{magnetostatic fields}
  {
    \begin{align*}
      \vec{B} &= \mu_{0} \frac{N}{L} I \hat{z} \quad \textrm{(solenoid)}\\
      \vec{B} &= \frac{\mu_{0} N I}{2 \pi r} \hat{\phi}\\
    \end{align*}
    There is no field outside a toroid or solenoid.
  }
\end{flashcard}

\begin{flashcard}{cyclotron}
  {
    \begin{align*}
      \vec{F} &= q \vec{v} \times \vec{B} = qvB \hat{x}\\
      R &= \frac{mv}{qB}\\
    \end{align*}
  }
\end{flashcard}

\begin{flashcard}{inductance}
  {
    \begin{align*}
      \mathcal{E} &= \int_{C} \vec{E} \cdot d\vec{l} = \frac{- \partial \Phi_{B}}{\partial t}\\
      \Phi_{21} &= M_{12} I_{1}\\
      \Phi_{B} &= L I\\
    \end{align*}
    Induced curents oppose changes in magnetic flux
  }
\end{flashcard}

\begin{flashcard}{inductance configurations}
  {
    \begin{align*}
      L &= \frac{\mu_{0 N^{2} A}}{l} \quad \textrm{solenoid}\\
    \end{align*}
  }
\end{flashcard}

\begin{flashcard}{multipoles}
  {
    \begin{align*}
      \vec{p} &= \int \vec{r} \rho(\vec{r}) d^{3}r\\
      V(\vec{r}) &= \frac{1}{4 \pi \epsilon_{0}} \frac{\vec{p} \cdot \vec{r}}{{\lvert \vec{r} \lvert}^{3}}\\
      \tau &= \vec{p} \times \vec{E}\\
      U &= - \vec{p} \cdot \vec{E}\\
      \vec{m} &= I \vec{A}\\
      \tau &= \vec{m} \times \vec{B}\\
      U &= - \vec{m} \times \vec{B}\\
    \end{align*}
  }
\end{flashcard}

\begin{flashcard}{em in media}
  {
    \begin{align*}
      \sigma_{b} &= \vec{P} \cdot \hat{n}\\
      \rho_{b} &= - \vec{\nabla} \cdot \vec{P}\\
      \epsilon &= \kappa \epsilon_{0}\\
    \end{align*}
  }
\end{flashcard}

\begin{flashcard}{em waves}
  {
    \begin{align*}
      \vec{E}(\vec{r}) &= E_{0} e^{i(\vec{k} \cdot \vec{r} - \omega t)} \hat{n}\\
      \vec{B}(\vec{r}) &= B_{0} e^{i(\vec{k} \cdot \vec{r} - \omega t)} (\hat{k} \times \hat{n})\\
      \vec{S} &= \frac{1}{\mu_{0}} (\vec{E} \times \vec{B}) = \frac{1}{2 \mu_{0}} \textrm{Re}[\vec{E} \times \vec{B}^{*}]\\
      I &= \langle S \rangle = \frac{c \epsilon_{0}}{2} E_{0}^{2}
    \end{align*}
  }
\end{flashcard}

\begin{flashcard}{radiation}
  {
    \begin{align*}
      P &= \frac{\mu_{0} q^{2} a^{2}}{6 \pi c} \quad \textrm{accelerating point charge}\\
      \langle S \rangle &= \frac{\mu_{0} p_{0}^{2} \omega^{4}}{32 \pi^{2} c} \frac{\sin^{2}(\theta)}{r^{2}}\\
      \langle P_{E} \rangle &= \frac{\mu_{0} p_{0}^{2} \omega^{4}}{12 \pi c}\\
      \langle P_{B} \rangle &= \frac{\mu_{0} m_{0}^{2} \omega^{4}}{12 \pi c^{3}}
    \end{align*}
    The bottom three equations are for a radiating dipole.
  }
\end{flashcard}

\begin{flashcard}{circuits}
  {
    \begin{align*}
      V_{R} &= IR\\
      V_{C} &= \frac{Q}{C}\\
      V_{L} &= \frac{L}{\ddot{Q}}\\
      R_{\textrm{series}} &= R_{i}\\
      \frac{1}{C_{\textrm{series}}} &= \frac{1}{C_{i}}\\
      L_{\textrm{series}} &= L_{i}\\
      \frac{1}{R_{\textrm{parallel}}} &= \frac{1}{R_{i}}\\
      C_{\textrm{parallel}} &= C_{i}\\
      \frac{1}{L_{\textrm{parallel}}} &= \frac{1}{L_{i}}\\
      R &= \frac{\rho l}{A}\\
      P &= IV\\
      U_{C} &= \frac{1}{2} C V^{2}\\
      U_{L} &= \frac{1}{2} L I^{2}\\
      I &= \frac{V}{R}(1 - e^{-\frac{t}{L/R}}) \quad \textrm{RL circuit}\\
      V &= V_{0}e^{-\frac{t}{RC}} \quad \textrm{RC circuit}\\
      \omega_{RLC} &= (L C)^{-\frac{1}{2}}
    \end{align*}
    \begin{itemize}
    \item circuit elements in series see same current
    \item circuit elements in parallel see same voltage
    \item current in/out of a node is conserved
    \item voltage is conserved over elements around a closed loop
    \end{itemize}
  }
\end{flashcard}

\begin{flashcard}{wave equation}
  {
    \begin{align*}
      \partial_{t}^{2} \psi &= v^{2} \partial_{x}^{2} \psi\\
      \psi(\vec{r}, t) &= A \cos(\vec{k} \cdot \vec{r} - \omega t + \phi)\\
      &= \textrm{Re}[A e^{i(\vec{k} \cdot \vec{r} - \omega t)}]
      v_{\textrm{phase}} &= \frac{\omega}{k}\\
      v_{\textrm{group}} &= \frac{\partial w}{\partial k}\\
      \frac{\omega}{k} &= \frac{c}{n} \ \lambda \rightarrow \frac{\lambda}{n} \textrm{light}\\
      \lambda &= \frac{2 \pi}{k}\\
      T &= \frac{2 \pi}{\omega}\\
      \omega &= 2 \pi f\\
    \end{align*}
  }
\end{flashcard}

\begin{flashcard}{wave configurations}
  {
    \begin{align*}
      v &= (\frac{T}{\mu})^{\frac{1}{2}} \ \textrm{string w/ tension $T$ and mass density $\mu$}\\
      v &= (\frac{\kappa}{\rho})^{\frac{1}{2}} \ \textrm{sound w/ bulk modulus $\kappa$ and density $\rho$}\\
    \end{align*}
  }
\end{flashcard}

\begin{flashcard}{polarization}
  {
    \begin{align*}
      I &= I_{0} \cos^{2}(\theta)\\
      \theta_{B} &= \arctan(\frac{n_{2}}{n_{1}})\\
    \end{align*}
  }
\end{flashcard}

\begin{flashcard}{interference}
  {
    \begin{align*}
      d \sin(\theta) &= m \lambda \ \textrm{max dub-slit}\\
      d \sin(\theta) &= (m + \frac{1}{2}) \lambda \ \textrm{min dub-slit}\\
      a \sin(\theta) &= m \lambda \ \textrm{min single-slit}\\
      D \sin(\theta) &= 1.22 \lambda \ \textrm{first circular diffraction minimum}
      d \sin(\theta) &= \frac{n \lambda}{2} \ \textrm{x-rays on crystal maxima}\\
      n_{2} > n_{1} &\implies \pi \ \textrm{phase shift}\\
      n_{2} < n_{1} &\implies 0 \ \textrm{phase shift}\\
    \end{align*}
  }
\end{flashcard}

\begin{flashcard}{geometric optics}
  {
    \begin{align*}
      \theta_{i} &= \theta_{r}
      n_{1} \sin(\theta_{1}) &= n_{2} \sin(\theta_{2})\\
      \frac{1}{s} + \frac{1}{s^{'}} &= \frac{1}{f}\\
      f &= \frac{R}{2} \ \textrm{spherical mirror}\\
      \frac{1}{f} &= (n - 1) (\frac{1}{R_{1}} - \frac{1}{R_{2}})\\
      m &= \frac{- s^{'}}{s}\\
    \end{align*}
  }
\end{flashcard}

\begin{flashcard}{misc optics}
  {
    \begin{align*}
      I &\propto I_{0} \lambda^{-4}a^{6}\\
      f &= (\frac{v + v_{r}}{v - v_{s}}) f_{0}\\
    \end{align*}
  }
\end{flashcard}


\end{document}
