\documentclass[avery5388, frame, grid]{flashcards}

\input{preamble}

\begin{document}

%% MECHANICS
\begin{flashcard}{kinematic equations}
  {\begin{align*}
    \bar{v} &= \frac{1}{2} (v_{i} + v_{f})\\
    \Delta v &= a \Delta t + v_{i}\\
    x(t) &= \frac{1}{2} a t^{2} + v_{i} t + x_{i}\\
    v_{f}^{2} - v_{i}^{2} &= 2 a \Delta x\\
  \end{align*}}
\end{flashcard}

\begin{flashcard}{uniform circular motion}
  {\begin{align*}
      a &= \frac{v^{2}}{r}\\
      v &= r \omega
  \end{align*}}
\end{flashcard}

\begin{flashcard}{kintec and potential energy}
  {
    \begin{align*}
      E_{\textrm{trans}} &= \frac{1}{2} m v^{2}\\
      E_{\textrm{rot}} &= \frac{1}{2} I \omega^{2}\\
      E_{\textrm{gpe}} &= m g h\\
      E_{\textrm{sho}} &= \frac{1}{2} k x^{2}
    \end{align*}
  }
\end{flashcard}

\begin{flashcard}{potential energy, force, work}
  {
    \begin{align*}
       dU &= -\vec{F} d\vec{l}\\
      \vec{F} &= - \vec{\nabla} U\\
      dW &= \vec{F} \cdot d\vec{l}
    \end{align*}
  }
\end{flashcard}

\begin{flashcard}{gravitational force}
  {
    \begin{align*}
      \vec{F} &= - G m_{1} m_{2} r^{-2} \hat{r_{12}}
    \end{align*}
  }
\end{flashcard}

\begin{flashcard}{angular momentum}
  {
    \begin{align*}
      \vec{L} &= \vec{r} \times \vec{p} = I \vec{\omega}
    \end{align*}
  }
\end{flashcard}

\begin{flashcard}{torque}
  {
    \begin{align*}
      \tau = \vec{r} \times \vec{F} = \frac{d \vec{L}}{dt}
    \end{align*}
  }
\end{flashcard}

\begin{flashcard}{constant angular velocity}
  {
    \begin{align*}
      \vec{F} = - m \Omega^{2} r \hat{r}_{\textrm{centrifuge}} - 2 m \vec{\Omega} \times \vec{v}\\
    \end{align*}
  }
\end{flashcard}

\begin{flashcard}{spherical elements}
  {
    \begin{align*}
      d\vec{l} &= dr \hat{r} + r d\theta \hat{\theta} + r \sin\theta d\phi \hat{\phi}\\
      d\vec{S} &= r^{2}\sin\theta d\theta d\phi \hat{r} + r \sin\theta dr d\phi \hat{\theta} + r dr d\theta \hat{\phi}\\
      dV &= r^{2}\sin\theta dr d\theta d\phi\\
    \end{align*}
  }
\end{flashcard}

\begin{flashcard}{moment of inertia}
  {
    \begin{align*}
      I &= \int{s^{2} dm}\\
      I &= I_{\textrm{CM}} + M s^{2}\\
      I_{z} &= I_{x} + I_{y} \; \textrm{for $z$ perpendicular to the body lying in the $x-y$ plane}\\
    \end{align*}
  }
\end{flashcard}

\begin{flashcard}{Lagrangian}
  {
    \begin{align*}
        L(\dot{q}_{i}, q_{i}, t) &= T - U\\
        \frac{d}{dt} \frac{\partial L}{\partial \dot{q}_{i}} &= \frac{\partial L}{\partial q_{i}}\\
        p_{i} &= \frac{\partial L}{\partial \dot{q}_{i}}\\
    \end{align*}
  }
\end{flashcard}

\begin{flashcard}{Hamiltonian}
  {
    \begin{align*}
      H(p, q) &= \sum_{i} p_{i} \dot{q}_{i} - L\\
      H &= T + U \; \textrm{if $U$ is independent of $\dot{q}$ and $t$}\\
      \dot{p} &= - \frac{\partial H}{\partial q}\\
      \dot{q} &= \frac{\partial H}{\partial p}\\
    \end{align*}
  }
\end{flashcard}

\begin{flashcard}{effective orbital potential}
  {
    \begin{align*}
      E = T + V &= \frac{1}{2} m \dot{r}^{2} + \frac{l^2}{2 m r^{2}} + U(r)\\
      l &= m r^{2} \dot{\phi}\\
      \mu &= \frac{m_{1} m_{2}}{m_{1} + m_{2}}\\
      E &> 0 \implies \textrm{hyperbolic}\\
      E &= 0 \implies \textrm{parabolic}\\
      E &< 0 \implies \textrm{elliptical}\\
      V^{'}(r) &= 0 \; \textrm{corresponds to circular orbit}\\
      V^{''}(r)& \; \textrm{determines stability}\\
    \end{align*}
  }
\end{flashcard}

\begin{flashcard}{Kepler's laws}
  {
    \begin{itemize}
    \item Planets move in elliptical orbits with one focus at the sun
    \item Equal orbital area sweeps out equal orbital time
    \item $T = ka^{\frac{3}{2}}$ for all planets
    \end{itemize}
  }
\end{flashcard}

\begin{flashcard}{damped oscillation}
  {
    \begin{align*}
      0 &= m\ddot{x} + b\dot{x} + kx \\
      \beta &= \frac{b}{2m}\\
      \omega_{0} &= (\frac{k}{m})^{\frac{1}{2}}\\
    \end{align*}
  }
  {
    \begin{itemize}
    \item $\beta^{2} > \omega_{0}^{2}$: overdamped decays exponentially
    \item $\beta^{2} = \omega_{0}^{2}$: critically damed
    \item $\beta^{2} < \omega_{0}^{2}$: underdamped decays exponentially, modulated by a sinusoid at $\omega_{1}^{2} = \omega_{0}^{2} - \beta^{2}$
    \end{itemize}
  }
\end{flashcard}

\begin{flashcard}{driven oscillation}
  {
    \begin{align*}
      \omega_{\textrm{R}} &= (\omega_{0}^{2} - 2 \beta^{2})^{\frac{1}{2}}\\
      D &\propto {\vert \omega_{0}^{2} - \omega^{2} \vert}^{-1} \; \textrm{for undamped, driven oscillator}\\
    \end{align*}
  }
\end{flashcard}

\begin{flashcard}{oscillators}
  {
    \begin{itemize}
    \item pendulum $\omega = (\frac{g}{l})^{\frac{1}{2}}$
    \item physical pendulum $\omega = (\frac{mgR}{I})^{\frac{1}{2}}$
    \end{itemize}
  }
\end{flashcard}

\begin{flashcard}{fluid dynamics}
  {
    \begin{align*}
      \frac{v^2}{2} &+ g z + \frac{p}{\rho} = \textrm{const}\\
      p \; \textrm{pressure, } &\rho \; \textrm{fluid density, } z \; \textrm{height of a point along streamline}\\
      F & = \rho V g\\
    \end{align*}
  }
\end{flashcard}

\begin{flashcard}{mechanics misc.}
  {
    \begin{itemize}
      \item an object falls over when its center of mass is not above its contact with the ground
    \end{itemize}
  }
\end{flashcard}


%% EM
\begin{flashcard}{Maxwell's equations}
  {
    \begin{align*}
      \vec{\nabla} \cdot \vec{E} &= \frac{\rho}{\epsilon_{0}}\\
      \vec{\nabla} \times \vec{E} &= \frac{- \partial \vec{B}}{\partial t}\\
      \vec{\nabla} \cdot \vec{B} &= 0\\
      \vec{\nabla} \times \vec{B} &= \mu_{0} \vec{J} + \mu_{0} \epsilon_{0} \frac{\partial \vec{E}}{\partial t}\\
      \frac{\partial \rho}{\partial t} &+ \vec{\nabla} \cdot \vec{J} = 0\\
      \vec{E} &= - \vec{\nabla} \phi - \frac{\partial \vec{A}}{\partial t}\\
      \vec{B} &= \vec{\nabla} \times \vec{A}\\
    \end{align*}
  }
\end{flashcard}

\begin{flashcard}{electrostatics}
  {
    \begin{align*}
      \vec{E} &= -\vec{\nabla} \phi \implies \phi = - \int_{a}^{b} \vec{E} \cdot d\vec{l}\\
      \nabla^{2} \phi &= - \frac{\rho}{\epsilon_{0}} \implies
      \phi = \frac{1}{4 \pi \epsilon_{0}} \int \frac{\rho(\vec{r}^{'})}{\lvert \vec{r} - \vec{r}^{'}\lvert} d^{3}\vec{r}^{'}\\
      \int_{S} \vec{E}(\vec{r}) \cdot d\vec{S} &= \frac{Q_{\textrm{enc}}}{\epsilon_{0}}\\
      \int_{C} \vec{E}(\vec{r}) \cdot d\vec{l} &= 0\\
    \end{align*}
  }
\end{flashcard}

\begin{flashcard}{boundary conditions}
  {
    Electrostatics
    \begin{align*}
      E^{\parallel}_{\textrm{out}} - E^{\parallel}_{\textrm{in}} &= 0\\
      E^{\bot}_{\textrm{out}} - E^{\bot}_{\textrm{in}} &= \frac{\sigma}{\epsilon_{0}}\\
      \phi &\textrm{is continuous}\\
      \partial \phi & \textrm{continuous when no surface charge is present}\\
    \end{align*}
    Magnetostatics
    \begin{align*}
      B^{\parallel}_{\textrm{out}} - B^{\parallel}_{\textrm{in}} &= \mu_{0} \vec{k} \times \hat{n}
      B^{\bot}_{\textrm{out}} - B^{\bot}_{\textrm{in}} &= 0
    \end{align*}
    $\vec{k}$ is the surface current density
  }
\end{flashcard}

\begin{flashcard}{conductors}
  {
    \begin{align*}
      \phi &= \textrm{const. throughout conductor}\\
      & \implies \textrm{E-field inside is zero}\\
      & \implies Q_{\textrm{inside}} = 0\\
      & \implies Q_{\textrm{net}} \textrm{confined to surface}\\
      & \implies \textrm{E-field outside is perpendicular to surface}\\
    \end{align*}
  }
\end{flashcard}

\begin{flashcard}{work and energy in (electro/magneto)statics}
  {
    \begin{align*}
      W &= \frac{1}{2} \int \rho \phi d^{3}r\\
      U_{E} &= \frac{\epsilon_{0}}{2} \int {\lvert \vec{E} \lvert}^{2} d^{3}r\\
      U_{B} &= \frac{1}{2 \mu_{0}} \int {\lvert \vec{B} \lvert}^{2} d^{3}r\\
    \end{align*}
  }
\end{flashcard}

\begin{flashcard}{capacitors}
  {
    \begin{align*}
      Q &= C \phi\\
      C_{\textrm{parallel plate}} &= \frac{\epsilon_{0} A}{d}\\
      U &= \frac{1}{2} Q^{2} C^{-1} = \frac{1}{2} C \phi^{2}\\
    \end{align*}
  }
\end{flashcard}

\begin{flashcard}{magnetostatics}
  {
    \begin{align*}
      \int_{S} \vec{B} \cdot d\vec{s} &= 0\\
      \int_{C} \vec{B} \cdot d\vec{l} &= \mu_{0} I_{\textrm{enc}}\\
      \vec{F_{B}} &= q\vec{v} \times \vec{B}\\
      \vec{B}(\vec{r}) &= \frac{\mu_{0} I}{4 \pi} \int \frac{d\vec{l} \times \vec{r^{'}}}{{\lvert \vec{r^{'}}\lvert}^{3}}\\
    \end{align*}
  }
\end{flashcard}

\begin{flashcard}{magnetostatic fields}
  {
    \begin{align*}
      \vec{B} &= \mu_{0} \frac{N}{L} I \hat{z} \quad \textrm{(solenoid)}\\
      \vec{B} &= \frac{\mu_{0} N I}{2 \pi r} \hat{\phi}\\
    \end{align*}
    There is no field outside a toroid or solenoid.
  }
\end{flashcard}

\begin{flashcard}{cyclotron}
  {
    \begin{align*}
      \vec{F} &= q \vec{v} \times \vec{B} = qvB \hat{x}\\
      R &= \frac{mv}{qB}\\
    \end{align*}
  }
\end{flashcard}

\begin{flashcard}{inductance}
  {
    \begin{align*}
      \mathcal{E} &= \int_{C} \vec{E} \cdot d\vec{l} = \frac{- \partial \Phi_{B}}{\partial t}\\
      \Phi_{21} &= M_{12} I_{1}\\
      \Phi_{B} &= L I\\
    \end{align*}
    Induced curents oppose changes in magnetic flux
  }
\end{flashcard}

\begin{flashcard}{inductance configurations}
  {
    \begin{align*}
      L &= \frac{\mu_{0 N^{2} A}}{l} \quad \textrm{solenoid}\\
    \end{align*}
  }
\end{flashcard}

\begin{flashcard}{multipoles}
  {
    \begin{align*}
      \vec{p} &= \int \vec{r} \rho(\vec{r}) d^{3}r\\
      V(\vec{r}) &= \frac{1}{4 \pi \epsilon_{0}} \frac{\vec{p} \cdot \vec{r}}{{\lvert \vec{r} \lvert}^{3}}\\
      \tau &= \vec{p} \times \vec{E}\\
      U &= - \vec{p} \cdot \vec{E}\\
      \vec{m} &= I \vec{A}\\
      \tau &= \vec{m} \times \vec{B}\\
      U &= - \vec{m} \times \vec{B}\\
    \end{align*}
  }
\end{flashcard}

\begin{flashcard}{em in media}
  {
    \begin{align*}
      \sigma_{b} &= \vec{P} \cdot \hat{n}\\
      \rho_{b} &= - \vec{\nabla} \cdot \vec{P}\\
      \epsilon &= \kappa \epsilon_{0}\\
    \end{align*}
  }
\end{flashcard}

\begin{flashcard}{em waves}
  {
    \begin{align*}
      \vec{E}(\vec{r}) &= E_{0} e^{i(\vec{k} \cdot \vec{r} - \omega t)} \hat{n}\\
      \vec{B}(\vec{r}) &= B_{0} e^{i(\vec{k} \cdot \vec{r} - \omega t)} (\hat{k} \times \hat{n})\\
      \vec{S} &= \frac{1}{\mu_{0}} (\vec{E} \times \vec{B}) = \frac{1}{2 \mu_{0}} \textrm{Re}[\vec{E} \times \vec{B}^{*}]\\
      I &= \langle S \rangle = \frac{c \epsilon_{0}}{2} E_{0}^{2}
    \end{align*}
  }
\end{flashcard}

\begin{flashcard}{radiation}
  {
    \begin{align*}
      P &= \frac{\mu_{0} q^{2} a^{2}}{6 \pi c} \quad \textrm{accelerating point charge}\\
      \langle S \rangle &= \frac{\mu_{0} p_{0}^{2} \omega^{4}}{32 \pi^{2} c} \frac{\sin^{2}(\theta)}{r^{2}}\\
      \langle P_{E} \rangle &= \frac{\mu_{0} p_{0}^{2} \omega^{4}}{12 \pi c}\\
      \langle P_{B} \rangle &= \frac{\mu_{0} m_{0}^{2} \omega^{4}}{12 \pi c^{3}}
    \end{align*}
    The bottom three equations are for a radiating dipole.
  }
\end{flashcard}

\begin{flashcard}{circuits}
  {
    \begin{align*}
      V_{R} &= IR\\
      V_{C} &= \frac{Q}{C}\\
      V_{L} &= \frac{L}{\ddot{Q}}\\
      R_{\textrm{series}} &= R_{i}\\
      \frac{1}{C_{\textrm{series}}} &= \frac{1}{C_{i}}\\
      L_{\textrm{series}} &= L_{i}\\
      \frac{1}{R_{\textrm{parallel}}} &= \frac{1}{R_{i}}\\
      C_{\textrm{parallel}} &= C_{i}\\
      \frac{1}{L_{\textrm{parallel}}} &= \frac{1}{L_{i}}\\
      R &= \frac{\rho l}{A}\\
      P &= IV\\
      U_{C} &= \frac{1}{2} C V^{2}\\
      U_{L} &= \frac{1}{2} L I^{2}\\
      I &= \frac{V}{R}(1 - e^{-\frac{t}{L/R}}) \quad \textrm{RL circuit}\\
      V &= V_{0}e^{-\frac{t}{RC}} \quad \textrm{RC circuit}\\
      \omega_{RLC} &= (L C)^{-\frac{1}{2}}
    \end{align*}
    \begin{itemize}
    \item circuit elements in series see same current
    \item circuit elements in parallel see same voltage
    \item current in/out of a node is conserved
    \item voltage is conserved over elements around a closed loop
    \end{itemize}
  }
\end{flashcard}


%% OPTICS WAVES
\begin{flashcard}{wave equation}
  {
    \begin{align*}
      \partial_{t}^{2} \psi &= v^{2} \partial_{x}^{2} \psi\\
      \psi(\vec{r}, t) &= A \cos(\vec{k} \cdot \vec{r} - \omega t + \phi)\\
      &= \textrm{Re}[A e^{i(\vec{k} \cdot \vec{r} - \omega t)}]
      v_{\textrm{phase}} &= \frac{\omega}{k}\\
      v_{\textrm{group}} &= \frac{\partial w}{\partial k}\\
      \frac{\omega}{k} &= \frac{c}{n} \ \lambda \rightarrow \frac{\lambda}{n} \textrm{light}\\
      \lambda &= \frac{2 \pi}{k}\\
      T &= \frac{2 \pi}{\omega}\\
      \omega &= 2 \pi f\\
    \end{align*}
  }
\end{flashcard}

\begin{flashcard}{wave configurations}
  {
    \begin{align*}
      v &= (\frac{T}{\mu})^{\frac{1}{2}} \ \textrm{string w/ tension $T$ and mass density $\mu$}\\
      v &= (\frac{\kappa}{\rho})^{\frac{1}{2}} \ \textrm{sound w/ bulk modulus $\kappa$ and density $\rho$}\\
    \end{align*}
  }
\end{flashcard}

\begin{flashcard}{polarization}
  {
    \begin{align*}
      I &= I_{0} \cos^{2}(\theta)\\
      \theta_{B} &= \arctan(\frac{n_{2}}{n_{1}})\\
    \end{align*}
  }
\end{flashcard}

\begin{flashcard}{interference}
  {
    \begin{align*}
      d \sin(\theta) &= m \lambda \ \textrm{max dub-slit}\\
      d \sin(\theta) &= (m + \frac{1}{2}) \lambda \ \textrm{min dub-slit}\\
      a \sin(\theta) &= m \lambda \ \textrm{min single-slit}\\
      D \sin(\theta) &= 1.22 \lambda \ \textrm{first circular diffraction minimum}
      d \sin(\theta) &= \frac{n \lambda}{2} \ \textrm{x-rays on crystal maxima}\\
      n_{2} > n_{1} &\implies \pi \ \textrm{phase shift}\\
      n_{2} < n_{1} &\implies 0 \ \textrm{phase shift}\\
    \end{align*}
  }
\end{flashcard}

\begin{flashcard}{geometric optics}
  {
    \begin{align*}
      \theta_{i} &= \theta_{r}
      n_{1} \sin(\theta_{1}) &= n_{2} \sin(\theta_{2})\\
      \frac{1}{s} + \frac{1}{s^{'}} &= \frac{1}{f}\\
      f &= \frac{R}{2} \ \textrm{spherical mirror}\\
      \frac{1}{f} &= (n - 1) (\frac{1}{R_{1}} - \frac{1}{R_{2}})\\
      m &= \frac{- s^{'}}{s}\\
    \end{align*}
    
    \begin{itemize}
    \item $s$, $s^{'}$ positive distances $\implies$ same side as light ray
    \item negative distances $\implies$ opposite side as light ray
    \item $f$ positive for converging lense, negative for diverging
      \item $f$ positive for concave mirror, negative for convex mirror
    \end{itemize}
  }
\end{flashcard}

\begin{flashcard}{misc optics}
  {
    \begin{align*}
      I &\propto I_{0} \lambda^{-4}a^{6}\\
      f &= (\frac{v + v_{r}}{v - v_{s}}) f_{0}\\
    \end{align*}
  }
\end{flashcard}

\begin{flashcard}{fourier transform and series}
  {
    \begin{align*}
      a_{n} &= \frac{2}{T} \int_{0}^{T} f(t) \cos{\frac{2 \pi n x}{T}}\\
      b_{n} &= \frac{2}{T} \int_{0}^{T} f(t) \sin{\frac{2 \pi n x}{T}}\\
      f(x) &= \frac{a_{0}}{2} + \sum_{n = 1}^{\infty} a_{n} \cos{\frac{2 \pi n x}{T}} + \sum_{n = 1}^{\infty} b_{n} \sin{\frac{2 \pi n x}{T}}\\
      c_{n} &= \frac{1}{T} \int_{0}^{T} f(t) \exp{\frac{-i 2 \pi n x}{T}}\\
      f(x) &= \sum_{n = -\infty}^{\infty} c_{n} \exp{\frac{i 2 \pi n x}{T}}\\
    \end{align*}
  }
\end{flashcard}

\begin{flashcard}{pipes}
  {
    \begin{align*}
      L &= \lambda(\frac{1}{4} + \frac{n}{2}) \; \textrm{half-open pipe}\\
    \end{align*}
  }
\end{flashcard}


%% STAT MECH
\begin{flashcard}{basic stat mech}
  {
    \begin{align*}
      p_{i} &= \frac{e^{- \beta E_{i}}}{Z}\\
      Z &= \sum_{j} e^{-\beta E_{j}}\\
      \beta &= (k_{B} T)^{-1}\\
      \langle O \rangle &= \sum_{i} p_{i} O_{i}\\
      \langle E \rangle &= - \partial_{\beta} \textrm{ln} Z\\
      Z_{N} &= \frac{1}{N! h^{3N}} \int e^{-\beta H(\vec{p}_{1},...\vec{p}_{n};\vec{x}_{1},...\vec{x}_{n})} d^{3}p_{1}...d^{3}p_{n} d^{3}x_{1}...d^{3}x_{n}\\
      k_{B} &= 1.380 649 e-23 \textrm{J} \textrm{K}^{-1}
    \end{align*}
  }
\end{flashcard}

\begin{flashcard}{entropy}
  {
    \begin{align*}
      S &= k_{B} \textrm{ln} \Omega\\
      S &= - k_{B} \sum_{i} p_{i} \textrm{ln} p_{i} = \partial_{T}(k_{B} T \textrm{ln} Z)\\
      S &= N k_{B} \textrm{ln} \frac{V T^{3/2}}{N} + \textrm{const.} \ \textrm{(monoatomic ideal gas)}
    \end{align*}
  }
\end{flashcard}

\begin{flashcard}{equipartition theorem}
  Each quadratic term (d.o.f.) in the Hamiltonian for a particle contributes $k_{B} T / 2$
  to the internal energy of a particle
\end{flashcard}

\begin{flashcard}{misc combinatorics}
  {
    \begin{align*}
      (N, M) &= \frac{N!}{(N - M)! M!}\\
      \textrm{ln} n! &= n \textrm{ln} n - n \ \textrm{$n$ large}
    \end{align*}
  }
\end{flashcard}

\begin{flashcard}{thermodynamic laws}
  \begin{itemize}
  \item Energy is neither created nor destoryed\\
    $\Delta U = Q - W$
  \item There does not exista  process in which the sole effect is to transfer heat from a
    body at lower temperature to a body at higher temperature\\
    $\Delta S \ge \int \frac{\delta Q}{T}$
  \item Entropy  is zero at absolute zero temperature. There is only one microstate at absolute
    zero temperature
  \end{itemize}
\end{flashcard}

\begin{flashcard}{ideal gas law}
  {
    \begin{align*}
      P V &= N k_{B} T
    \end{align*}
  }
\end{flashcard}

\begin{flashcard}{reversible process}
  \begin{itemize}
  \item system is in equilibrium at every instant
  \item $\delta W = P dV$
  \item $\delta Q = T dS$
  \end{itemize}
\end{flashcard}

\begin{flashcard}{quasistatic}
  \begin{itemize}
  \item in thermal equilibrium at every instant, but not necessarily reversible
  \end{itemize}
\end{flashcard}

\begin{flashcard}{adiabatic}
  \begin{itemize}
  \item $\delta Q = 0$
  \end{itemize}
\end{flashcard}

\begin{flashcard}{isentropic / reversible adiabatic}
  \begin{itemize}
  \item reversible and adiabatic
  \item $\iff \Delta S = 0$
  \item $PV^{\gamma} = \textrm{const.} \ ; \gamma = \frac{C_{P}}{C_{V}}$
  \end{itemize}
\end{flashcard}

\begin{flashcard}{IsoX}
  \begin{itemize}
  \item X held constant
  \end{itemize}
\end{flashcard}

\begin{flashcard}{free expansion}
  \begin{itemize}
  \item $\Delta T = 0$
  \item irreversible
  \item $P V = P^{'} V^{'}$
  \end{itemize}
\end{flashcard}

\begin{flashcard}{thermodynamic identities}
  {
    \begin{align*}
      dU &= T dS - P dV\\
      T &= \frac{\partial U}{\partial S} \rvert_{V}\\
      P & = - \frac{\partial U}{\partial V} \rvert_{S}\\
      \frac{\partial P}{\partial S} \rvert_{V} &= - \frac{\partial T}{\partial V} \rvert_{S}
    \end{align*}
  }
\end{flashcard}

\begin{flashcard}{heat capacity}
  {
    \begin{align*}
      C_{V} &= \frac{\partial Q}{\partial T} \rvert_{V} = \frac{\partial U}{\partial T}\\
      C_{P} &= \frac{\partial Q}{\partial T} \rvert_{P}\\
      C_{P} - C_{V} &= N k_{B} \ \textrm{(ideal gas)}\\
      Q &= m c \Delta T\\
      c_{\textrm{water}} &= 4.18 \ \textrm{J} \textrm{K}^{-1} \textrm{g}^{-1}
    \end{align*}
  }
\end{flashcard}

\begin{flashcard}{model systems}
  {
    \begin{align*}
      e &= 1 - \lvert \frac{Q_{C}}{Q_{H}} \rvert\\
      e &= 1 - \frac{T_{C}}{T_{H}} \ \textrm{theoretical max, carnot}
    \end{align*}
  }
  \begin{itemize}
    \item use thermodynamic identities to calculate areas
    \item clockwise paths in $P$ - $V$ and $T$ - $S$ planes do positive work
  \end{itemize}
\end{flashcard}

\begin{flashcard}{monoatomic ideal gas}
  {
    \begin{align*}
      Z_{N} &= \frac{V^{N}}{N! h^{3N}}(2 \pi m k_{B} T)^{3^{N}/2}\\
      U &= \frac{3}{2} N k_{B} T\\
      v_{\textrm{rms}} &= (\frac{3 k_{B} T}{m})^{1/2}\\
      K &= \gamma P
    \end{align*}
  }
\end{flashcard}

\begin{flashcard}{quantum stat mech}
  {
    \begin{align*}
      F_{\textrm{FD}}(E_{i}) &= \frac{1}{e^{(E_{i} - \mu) / k_{B} T} + 1}\\
      F_{\textrm{BE}}(E_{i}) &= \frac{1}{e^{(E_{i} - \mu) / k_{B} T} - 1}\\
      \langle N \rangle &= \sum_{i} g(E_{i}) F(E_{i})\\
    \end{align*}
  }
\end{flashcard}


%% QUANTUM
\begin{flashcard}{schroedinger equation}
  {
    \begin{align*}
      \Psi(x, t) &= e^{-iE_{n}t/\hbar} \psi_{n}(x) \; \textrm{TDSE}\\
    \end{align*}
    
    \begin{itemize}
    \item $\psi_{n}$ are orthogonal
    \item $\psi$ continuous, $\psi^{'}$ continuous except when $V(x) = \infty$
    \item $\psi$ can be taken to be real, momentum expectation value vanishes for singleton state
    \item ground state has no nodes, $n^{\textrm{th}}$ excited state has $n$ nodes
    \item for an even potential $\psi_{n}$ is even for $n$ even, and odd for $n$ odd
    \end{itemize}
  }
\end{flashcard}

\begin{flashcard}{uncertainty principle}
  {
    \begin{align*}
      \sigma_{A}^{2} \sigma_{B}^{2} &\ge (\frac{1}{2i} \langle [A, B] \rangle)^{2}\\
      \sigma_{A}^{2} &= \langle A^{2} \rangle - {\langle A \rangle}^{2}
    \end{align*}
  }
\end{flashcard}

\begin{flashcard}{quantum configurations}
  {
    \begin{align*}
      E &= \frac{j(j + 1)\hbar^{2}}{2 I} \; \textrm{rigid rotator}\\
      E_{n} &= \frac{\hbar^{2} \pi^{2} n^{2}}{2 m a^{2}} \; \ket{\psi_{n}} = (\frac{2}{a})^{1/2} \sin(\frac{n \pi x}{a}) \; \textrm{infinite square well}\\
      \ket{\psi} &= e^{\pm i k x} \; E = \frac{\hbar^{2} k^{2}}{2 m} \; p = \hbar k \; \textrm{free particle}\\
      &\textrm{free particle is sum of these wave functions, particle w/ definite momentum is not normalizable, i.e. doesn't exist}\\
      \ket{\psi} &= (\frac{m \alpha}{\hbar^{2}})^{1/2} e^{-m \alpha \lvert x \rvert / \hbar^{2}} \; E = \frac{-m \alpha^{2}}{2 \hbar^{2}} \; \textrm{delta well}\\
      &\textrm{The finite square well is symmetric about zero so the states have definite parity.The ground state is even.}\\
    \end{align*}
  }
\end{flashcard}

\begin{flashcard}{QHO}
  {
    \begin{align*}
      H &= \hbar \omega (a^{\dagger}a + \frac{1}{2}) \; [a, a^{\dagger}] = 1 \; \textrm{QHO}\\
      &\textrm{ground state of QHO is gaussian}\\
      \langle T \rangle &= \langle V \rangle = \frac{E_{n}}{2} \; \textrm{QHO virial theorem}\\
      a^{\dagger} \ket{n} &= (n + 1)^{1/2} \ket{n + 1}\\
      a \ket{n} &= (n)^{1/2} \ket{n - 1}\\
      E_{n} &= \hbar \omega (N + \frac{3}{2}) \; N = n_{1} + n_{2} + n_{3} \; \textrm{QHO3D}
    \end{align*}
  }
\end{flashcard}

\begin{flashcard}{QM in 3D}
  {
    \begin{align*}
      \frac{-h^{2}}{2m} \partial_{r}^{2} u + (V + \frac{h^{2}}{2m} \frac{l(l+1)}{r^{2}}) u &= E u\\
      u(r) &= r R(r)\\
      \int_{0}^{\infty} {\lvert R(r) \rvert}^{2} r^{2} dr &= 1
    \end{align*}
  }
\end{flashcard}

\begin{flashcard}{angular momentum}
  {
    \begin{align*}
      [L_{i}, L_{j}] &= \epsilon^{ijk} i \hbar L_{k}\\
      [L^{2}, L_{i}] &= 0\\
      L_{z} Y_{l}^{m} &= m \hbar Y_{l}^{m}\\
      L^{2} Y_{l}^{m} &= \hbar^{2} l(l + 1) Y_{l}^{m}\\
    \end{align*}
  }
\end{flashcard}

\begin{flashcard}{spin operators}
  {
    \begin{align*}
      S_{\chi} &= S_{\chi}^{(1)} + \dots + S_{\chi}^{(n)}\\
      S_{\chi}^{(m)} &= I \odot ... \odot S_{\chi} \odot ... \odot I\\
      x_{1} &= 2^{-1/2} \begin{pmatrix} 1\\ 1 \end{pmatrix} \; x_{2} = 2^{-1/2} \begin{pmatrix} 1 -1 \end{pmatrix}\\
      y_{1} &= 2^{-1/2} \begin{pmatrix} 1\\ i \end{pmatrix} \; y_{2} = 2^{-1/2} \begin{pmatrix} 1 -i \end{pmatrix}\\
      \ket{\Psi} &= \ket{\psi} \ket{\chi}\\
      s_{\textrm{tot}} &= s + s^{'}, s + s^{'} - 1, \dots, \lvert s - s^{'} \rvert\\
      m_{\textrm{tot}} &= m_{s} + m_{s^{'}}
    \end{align*}

    \begin{itemize}
    \item add multiple spins by adding pairs
    \item boson (fermions) symmetric (anti-symmetric) under exchange of any two particles
    \item when adding spin-$1/2$ particles, spin states $s = n / 2$ are always symmetric
    \item 2e helium ground state is spatially symmetric
    \end{itemize}
  }
\end{flashcard}

\begin{flashcard}{hydrogen atom}
  {
    \begin{align*}
      a &= \frac{4 \pi \epsilon_{0} \hbar^{2}}{\mu e^{2}}\\
      \psi_{1} &\propto e^{-r/a}\\
      E_{n} &= \frac{- \hbar^{2}}{2 \mu a^{2}} \frac{1}{n^{2}} = \frac{-13.6 \textrm{ev}}{n^{2}}\\
      \alpha &= \frac{e^{2}}{4 \pi \epsilon_{0} \hbar c} \sim 1 / 137\\
      \psi \lvert_{r = 0} &= 0 \; \textrm{for} \; l \neq 0
    \end{align*}
  }
\end{flashcard}

\begin{flashcard}{approximation methods}
  {
    \begin{align*}
      E_{n} &= E_{n}^{0} + \braket{\psi_{n}^{0} \lvert H_{1} \rvert}{\psi_{n}^{0}}\\
      E_{n} &= E_{n}^{0} + \sum_{m \neq n} \frac{\braket{\psi_{m}^{0} \lvert H_{1} \rvert \psi_{n}^{0}}}{E_{n}^{0} - E_{m}^{0}}
    \end{align*}

    \begin{itemize}
    \item diagonalize expectation values of degenerate states with perturbation,
      eigenvalues are first order corrections
    \end{itemize}
  }
\end{flashcard}

\begin{flashcard}{Bohr model}
  {
    \begin{itemize}
    \item Electron moves in circular oribt. angular momentum is quantized $L = \hbar n$
    \item no electron radiation in a given shell (experimentally shown)
    \item mathces hydrogen transitions
    \item hydrogen energy $\propto \alpha^{2} m_{e} c^{2}$
    \end{itemize}
  }
\end{flashcard}

\begin{flashcard}{fine structure}
  {
    \begin{itemize}
    \item spin-orbit coupling and relativistic momentum
    \item energy shift $\propto \alpha^{4} m_{e} c^{2}$
    \item energy levels get j dependence, $m_{j}$ conserved, $l$ degeneracy broken
    \item $J_{2} = L^{2} + S^{2} + 2 L \dot S$
    \end{itemize}
  }
\end{flashcard}

\begin{flashcard}{Lamb shift}
  {
    \begin{itemize}
    \item QED
    \item energy shift $\propto \alpha^{5} m_{e} c^{2}$
    \item splits s and 2p levels with $j = \frac{1}{2}$ degeneracy
    \end{itemize}
  }
\end{flashcard}

\begin{flashcard}{hyperfine structure}
  {
    \begin{itemize}
    \item nucleus-electron spin-spin coupling
    \item energy shift $\propto \frac{m_{e}}{m_{p}} \alpha^{4} m_{e} c^{2}$
      \item ground state of hydrogen split depending on singlet or triplet configuration
    \end{itemize}
  }
\end{flashcard}

\begin{flashcard}{shell model}
  {
    \begin{itemize}
    \item s, p, d, f $\iff l = 0, 1, 2, 3, \dots$
    \item $2^{2}$ orbitals in each shell. $2(2l + 1)$ states in each orbital
    \item 1s, 2s, 2p, 3s, 3p, 3d
    \item shells fill in order. noble gases have filled shells
    \end{itemize}
  }
\end{flashcard}

\begin{flashcard}{stark effect}
  {
    \begin{itemize}
    \item $H_{1} = q \vec{E} \vec{x}$
    \item no change to ground state energy of hydrogenic atom to first order in $\lvert \vec{E} \rvert$
    \item n = 2, m = 0 states are split
    \item $\Delta E \propto q \lvert \vec{E} \rvert a_{0}$
    \end{itemize}
  }
\end{flashcard}

\begin{flashcard}{Zeeman effect}
  {
    \begin{itemize}
    \item $H_{1} = \frac{q}{2m} (\vec{L} + 2\vec{S}) \cdot \vec{B}$
    \item $\lvert \vec{B} \rvert$ small, Zeeman perturbs fine. $j, l, m_{j}$ quantum numbers. $j$ splits based on $m_{j}$, spin wants to be anti-aligned with $B$-field
      \item $\lvert \vec{B} \rvert$ large, fine perturbs Zeeman. $l, m_{l}, m_{s}$. l splits on $m_{l}, m_{s}$
    \end{itemize}
  }
\end{flashcard}

\begin{flashcard}{electric dipole radiation selection rules}
  {
    \begin{itemize}
    \item electric dipole approximation $\lambda \gg a \; \implies$ spatial variation of the field is negligible
    \item $\Delta m = \pm 1, 0$
    \item $\Delta l = \pm 1$
    \end{itemize}
  }
\end{flashcard}

\begin{flashcard}{blackbody radiation}
  {
    \begin{align*}
      I(\omega) \propto \frac{h \omega^{3}}{c^{2}} \frac{1}{e^{\hbar \omega / k_{B} T} - 1}\\
      \frac{dP}{dA} \propto T^{4}\\
      \lambda_{\textrm{peak}} &= 3 \cdot 10^{-3} K m T^{-1}
    \end{align*}
  }
\end{flashcard}

\begin{flashcard}{quantum misc.}
  {
    \begin{align*}
      \lambda_{\textrm{compton}} &= \frac{h}{mc}
    \end{align*}
  }
\end{flashcard}


%% RELATIVITY
\begin{flashcard}{realtivity basics}
  {
    \begin{align*}
      \beta &= \frac{v}{c}\\
      \gamma &= (1 - \beta^{2})^{-1/2}\\
      w &= \frac{v + u}{1 + vu/c^{2}}\\
      x^{\mu} &= \begin{pmatrix} ct\\ \vec{x} \end{pmatrix} \; p^{\mu} = \begin{pmatrix} \frac{E}{c}\\ \vec{p} \end{pmatrix}\\
      \vec{p} &= \gamma m \vec{v}\\
      \Lambda &= \begin{pmatrix} \gamma & -\gamma \beta\\ -\gamma \beta & \gamma \end{pmatrix}\\
      \lambda^{'} &= \lambda (\frac{1 + \Delta \beta}{1 - \Delta \beta})^{1/2}
    \end{align*}
  }
\end{flashcard}


\begin{flashcard}{relativity misc.}
  {
    \begin{itemize}
    \item $p^{\mu}p_{\mu} = m^{2}c^{2}$
    \item timelike $\Delta x^{\mu} \Delta x_{\mu} > 0$, there exists a rest frame where events occur at the same place
    \item spacelike $\Delta x^{\mu} \Delta x_{\mu} < 0$, there exists a rest frame where events occur at the same time
    \item lightlike $\Delta x^{\mu} \Delta x_{\mu} = 0$ rest frame is spaceship traveling between events
    \end{itemize}
  }
\end{flashcard}


%% LAB
\begin{flashcard}{graph reading}
  {
    \begin{itemize}
    \item straight line on log-log is $y = ax^{b}$ where $b$ is the slope
    \item straight line on log-linear is $y = C a^{bx}$ where $b \log_{10}(a)$ is the slope
    \item straight line on linear-log is $y = C\log_{a}(bx)$ where $C / \log_{10}(a)$ is the slope
    \end{itemize}
  }
\end{flashcard}

\begin{flashcard}{statistics}
  {
    \begin{align*}
      \sigma_{s}^{2} &= \frac{1}{n - 1} \sum_{i = 1}^{n} (x_{i} - \bar{x})^{2} \; \textrm{sample variance, use $1/n$ for pop}\\
      \delta(a A) &= a \delta A\\
      \delta(A + B) &= ((\delta A)^{2} + (\delta B)^{2})^{1/2}\\
      \delta(AB^{-1}) & = \delta(AB) = AB ((\frac{\delta A}{A})^{2} + (\frac{\delta B}{B})^{2})^(1/2)\\
      X &= \frac{\sum_{i} x_{i} \sigma_{x_{i}}^{-2}}{\sum_{i} \sigma_{x_{i}}^{-2}}\\
      \sigma^{2} &= \frac{1}{\sum_{i} \sigma_{x_{i}}^{-2}}\\
      P(n) &= \frac{\lambda^{n}e^{- \lambda}}{n!}\\
      \sigma &= N^{1/2} \; \textrm{Poisson distribution with $N$ large}\\
      P(t) &= \lambda e^{- \lambda t} \; \textrm{time between events}
    \end{align*}
  }
\end{flashcard}

\begin{flashcard}{electronics}
  {
    \begin{align*}
      V &= V_{0}e^{i \omega t}\\
      V &= I Z\\
      Z_{\textrm{cap}} &= \frac{1}{i \omega C}\\
      Z_{\textrm{ind}} &= i \omega L\\
      Z_{\textrm{res}} &= R\\
      Z_{\textrm{series}} &= \sum_{i} Z_{i}\\
      Z_{\textrm{parallel}}^{-1} &= \sum_{i} Z_{i}^{-1}
    \end{align*}

    \begin{itemize}
    \item resonant frequency is where $Im(Z) = 0$
    \item diode - current only flows in one direction once bias voltage is met -|>|-
    \item op-amp - output voltage is proportional to difference between inputs =|>-
    \end{itemize}
  }
\end{flashcard}

\begin{flashcard}{radiation detection}
  {
    \begin{itemize}
    \item nuclei are stopped faster than electrons alpha (He$^{4}$) $\propto 10^{-5}$m, $e \propto 10^{-3}$m
    \item nuceli interact w/ electrons. electrons interact w/ electrons or nuclei
    \item nuclei tend to have straight paths, electrons bounce around
    \item nuclei lose energy due to collisions, electrons lose energy by collision or bremsstrahlung radiation (photon emission due to deceleration)
    \end{itemize}
  }
\end{flashcard}

\begin{flashcard}{photon interactions}
  {
    \begin{itemize}
    \item photon absorbption
      \subitem dominant at a few keV
      \subitem photon absorbed by atom and electron released
      \subitem $E_{max} = E \gamma - \phi$ work function used for photon on material
    \item compton scattering
      \subitem photon scatters elastically off atomic electron, scattered electron is ejected from atom
      \subitem dominant for 10s of keV to a few Mev
      \subitem $\Delta \lambda = \frac{h}{mc}(1 - \cos(\theta))$
    \item pair production
      \subitem If $E_{\gamma} > 2m_{e}c^{2}$, dominant process for tens of MeV
      \subitem photon produces electron-positron pair
    \item decay channels compound like $\tau^{-1} = \tau^{-1}_{1} + \tau^{-1}_{2}$
    \end{itemize}
  }
\end{flashcard}

\begin{flashcard}{lasers}
  {
    \begin{itemize}
    \item 3-level $\Delta E_{01}$ small, 4-level $\Delta E_{01}$ large
    \item solid-state laser
      \subitem crystal, glass, transitions between atomic energy levels Nd:YAG, YAG E-field splits Nd
    \item collisional gas laser
      \subitem gas excited by KE from collisions, light filtered with conducting cavity
    \item molecular gas laser
      \subitem vibrational energy levels
    \item dye laser
      \subitem organic dye dissolved in water, electron transport chain
    \item semiconductor (dipole) laser
      \subitem excites semiconductor conduction band energy due to e-hole annihilation
    \item free electron laser
      \subitem accelerate with E-field
    \item michelson-morley interferometer $d \sin_{\theta} = m \lambda$
    \end{itemize}
  }
\end{flashcard}

\begin{flashcard}{lab misc.}
  {
    \begin{itemize}
    \item a confidence interval is a range of values for a parameter (typically the mean),
      associated to a confidence level. The confidence level gives the reliability of the estimation
      procedure, not to be confused with the long run probability of measuring the parameter
      within the interval. The upper limit on the parameter is the value
      at which the statement doesn't hold that the long run probability of measuring the parameter
      within the interval.
    \end{itemize}
  }
\end{flashcard}


%% MISC
\begin{flashcard}{fermi gas}
  {
    \begin{align*}
      E &= \frac{\hbar^{2}}{2m}(3 \frac{Nd}{V} \pi^{2})^{2/3}\\
      P &= \frac{(3\pi^{2})^{2/3}}{5m}(\frac{Nd}{V})^{5/3}\\
      T &= \frac{E}{k_{B}}\\
      p &= (2mE)^{1/2}
    \end{align*}
  }
\end{flashcard}

\begin{flashcard}{misc misc.}
  {
    \begin{align*}
      z_{\textrm{redshift}} &= \frac{\lambda_{\textrm{observed} - \lambda_{\textrm{emitted}}}}{\lambda{\textrm{emitted}}}
    \end{align*}
    
    \begin{itemize}
    \item p-type junction: electron holes, positive net charge
    \item n-type junction: no electron holes, negative net charge
    \end{itemize}
  }
\end{flashcard}

\end{document}
