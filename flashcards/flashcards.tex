\documentclass[avery5388, frame, grid]{flashcards}

\input{preamble}

\begin{document}

\begin{flashcard}{kinematic equations}
  {\begin{align*}
    \bar{v} &= \frac{1}{2} (v_{i} + v_{f})\\
    \Delta v &= a \Delta t + v_{i}\\
    x(t) &= \frac{1}{2} a t^{2} + v_{i} t + x_{i}\\
    v_{f}^{2} - v_{i}^{2} &= 2 a \Delta x\\
  \end{align*}}
\end{flashcard}

\begin{flashcard}{uniform circular motion}
  {\begin{align*}
      a &= \frac{v^{2}}{r}\\
  \end{align*}}
\end{flashcard}

\begin{flashcard}{kintec and potential energy}
  {
    \begin{align*}
      E_{\textrm{trans}} &= \frac{1}{2} m v^{2}\\
      E_{\textrm{rot}} &= \frac{1}{2} I \omega^{2}\\
      E_{\textrm{gpe}} &= m g h\\
      E_{\textrm{sho}} &= \frac{1}{2} k x^{2}\\
    \end{align*}
  }
\end{flashcard}

\begin{flashcard}{potential energy, force, work}
  {
    \begin{align*}
       dU &= -\vec{F} d\vec{l}\\
      \vec{F} &= - \vec{\nabla} U\\
      dW &= \vec{F} \cdot d\vec{l}\\
    \end{align*}
  }
\end{flashcard}

\begin{flashcard}{gravitational force}
  {
    \begin{align*}
      \vec{F} &= - G m_{1} m_{2} r^{-2} \hat{r_{12}}\\
    \end{align*}
  }
\end{flashcard}

\begin{flashcard}{angular momentum}
  {
    \begin{align*}
      \vec{L} &= \vec{r} \times \vec{p} = I \vec{\omega}\\
    \end{align*}
  }
\end{flashcard}

\begin{flashcard}{torque}
  {
    \begin{align*}
      \tau = \vec{r} \times \vec{F} = \frac{d \vec{L}}{dt}
    \end{align*}
  }
\end{flashcard}

\begin{flashcard}{constant angular velocity}
  {
    \begin{align*}
      \vec{F} = - m \Omega^{2} r \hat{r}_{\textrm{centrifuge}} - 2 m \vec{\Omega} \times \vec{v}
    \end{align*}
  }
\end{flashcard}

\begin{flashcard}{spherical elements}
  {
    \begin{align*}
      d\vec{l} &= dr \hat{r} + r d\theta \hat{\theta} + r \sin\theta d\phi \hat{\phi}\\
      d\vec{S} &= r^{2}\sin\theta d\theta d\phi \hat{r} + r \sin\theta dr d\phi \hat{\theta} + r dr d\theta \hat{\phi}\\
      dV &= r^{2}\sin\theta dr d\theta d\phi\\
    \end{align*}
  }
\end{flashcard}

\begin{flashcard}{moment of inertia}
  {
    \begin{align*}
      I &= \int{s^{2} dm}\\
      I &= I_{\textrm{CM}} + M s^{2}\\
      I_{z} &= I_{x} + I_{y} \; \textrm{for $z$ perpendicular to the body lying in the $x-y$ plane}\\
    \end{align*}
  }
\end{flashcard}

\begin{flashcard}{Lagrangian}
  {
    \begin{align*}
        L(\dot{q}_{i}, q_{i}, t) &= T - U\\
        \frac{d}{dt} \frac{\partial L}{\partial \dot{q}_{i}} &= \frac{\partial L}{\partial q_{i}}\\
        p_{i} &= \frac{\partial L}{\partial \dot{q}_{i}}\\
    \end{align*}
  }
\end{flashcard}

\begin{flashcard}{Hamiltonian}
  {
    \begin{align*}
      H(p, q) &= \sum_{i} p_{i} \dot{q}_{i} - L\\
      H &= T + U \; \textrm{if $U$ is independent of $\dot{q}$ and $t$}\\
      \dot{p} &= - \frac{\partial H}{\partial q}\\
      \dot{q} &= \frac{\partial H}{\partial p}\\
    \end{align*}
  }
\end{flashcard}

\begin{flashcard}{effective orbital potential}
  {
    \begin{align*}
      E = T + V &= \frac{1}{2} m \dot{r}^{2} + \frac{l^2}{2 m r^{2}} + U(r)\\
      l &= m r^{2} \dot{\phi}\\
      \mu &= \frac{m_{1} m_{2}}{m_{1} + m_{2}}\\
      E &> 0 \implies \textrm{hyperbolic}\\
      E &= 0 \implies \textrm{parabolic}\\
      E &< 0 \implies \textrm{elliptical}\\
    \end{align*}
  }
\end{flashcard}

\begin{flashcard}{Kepler's laws}
  {
    \begin{itemize}
    \item Planets move in elliptical orbits with one focus at the sun
    \item Equal orbital area sweeps out equal orbital time
    \item $T = ka^{\frac{3}{2}}$ for all planets
    \end{itemize}
  }
\end{flashcard}

\begin{flashcard}{damped oscillation}
  {
    \begin{align*}
      0 &= m\ddot{x} + b\dot{x} + kx \\
      \beta &= \frac{b}{2m}\\
      \omega_{0} &= (\frac{k}{m})^{\frac{1}{2}}\\
    \end{align*}
  }
  {
    \begin{itemize}
    \item $\beta^{2} > \omega_{0}^{2}$: overdamped decays exponentially
    \item $\beta^{2} = \omega_{0}^{2}$: critically damed
    \item $\beta^{2} < \omega_{0}^{2}$: underdamped decays exponentially, modulated by a sinusoid at $\omega_{1}^{2} = \omega_{0}^{2} - \beta^{2}$
    \end{itemize}
  }
\end{flashcard}

\begin{flashcard}{driven oscillation}
  {
    \begin{align*}
      \omega_{\textrm{R}} &= (\omega_{0}^{2} - 2 \beta^{2})^{\frac{1}{2}}\\
      D &\propto {\vert \omega_{0}^{2} - \omega^{2} \vert}^{-1} \; \textrm{for undamped, driven oscillator}\\
    \end{align*}
  }
\end{flashcard}

\begin{flashcard}{oscillators}
  {
    \begin{itemize}
    \item pendulum $\omega = (\frac{g}{l})^{\frac{1}{2}}$
    \item physical pendulum $\omega = (\frac{mgR}{I})^{\frac{1}{2}}$
    \end{itemize}
  }
\end{flashcard}

\begin{flashcard}{fluid dynamics}
  {
    \begin{align*}
      \frac{v^2}{2} &+ g z + \frac{p}{\rho} = \textrm{const}\\
      p \; \textrm{pressure, } &\rho \; \textrm{fluid density, } z \; \textrm{height of a point along streamline}\\
      F & = \rho V g\\
    \end{align*}
  }
\end{flashcard}

\begin{flashcard}{Maxwell's equations}
  {
    \begin{align*}
      \vec{\nabla} \cdot \vec{E} &= \frac{\rho}{\epsilon_{0}}\\
      \vec{\nabla} \times \vec{E} &= \frac{- \partial \vec{B}}{\partial t}\\
      \vec{\nabla} \cdot \vec{B} &= 0\\
      \vec{\nabla} \times \vec{B} &= \mu_{0} \vec{J} + c^{-2} \frac{\partial \vec{E}}{\partial t}\\
      \frac{\partial \rho}{\partial t} &+ \vec{\nabla} \cdot \vec{J} = 0\\
      \vec{E} &= - \vec{\nabla} \phi - \frac{\partial \vec{A}}{\partial t}\\
      \vec{B} &= \vec{\nabla} \times \vec{A}\\
    \end{align*}
  }
\end{flashcard}

\begin{flashcard}{electrostatics}
  {
    \begin{align*}
      \vec{E} &= -\vec{\nabla} \phi \implies \phi = - \int_{a}^{b} \vec{E} \cdot d\vec{l}\\
      \nabla^{2} \phi &= - \frac{\rho}{\epsilon_{0}} \implies
      \phi = \frac{1}{4 \pi \epsilon_{0}} \int \frac{\rho(\vec{r}^{'})}{\lvert \vec{r} - \vec{r}^{'}\lvert} d^{3}\vec{r}^{'}\\
      \int_{S} \vec{E}(\vec{r}) \cdot d\vec{S} &= \frac{Q_{\textrm{enc}}}{\epsilon_{0}}\\
      \int_{C} \vec{E}(\vec{r}) \cdot d\vec{l} &= 0\\
    \end{align*}
  }
\end{flashcard}

\begin{flashcard}{boundary conditions}
  {
    \begin{align*}
      E^{\parallel}_{\textrm{out}} - E^{\parallel}_{\textrm{in}} &= 0\\
      E^{\bot}_{\textrm{out}} - E^{\bot}_{\textrm{in}} &= \frac{\sigma}{\epsilon_{0}}\\
      \phi &\textrm{is continuous}\\
      \partial \phi & \textrm{continuous when no surface charge is present}\\
    \end{align*}
  }
\end{flashcard}

\begin{flashcard}{conductors}
  {
    \begin{align*}
      \phi &= \textrm{const. throughout conductor}\\
      & \implies \textrm{E-field inside is zero}\\
      & \implies Q_{\textrm{inside}} = 0\\
      & \implies Q_{\textrm{net}} \textrm{confined to surface}\\
      & \implies \textrm{E-field outside is perpendicular to surface}\\
    \end{align*}
  }
\end{flashcard}

\begin{flashcard}{work and energy in electrostatics}
  {
    \begin{align*}
      W &= \frac{1}{2} \int \rho \phi d^{3}r\\
      U_{E} &= \frac{\epsilon_{0}}{2} \int {\lvert \vec{E} \lvert}^{2} d^{3}r\\
    \end{align*}
  }
\end{flashcard}

\begin{flashcard}{capacitors}
  {
    \begin{align*}
      Q &= C \phi\\
      C_{\textrm{parallel plate}} &= \frac{\epsilon_{0} A}{d}\\
      U &= \frac{1}{2} Q^{2} C^{-1} = \frac{1}{2} C \phi^{2}\\
    \end{align*}
  }
\end{flashcard}

\end{document}
